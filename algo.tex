\documentclass[12pt]{report}
\usepackage{fontspec}
\usepackage[frenchb]{babel}
\usepackage{mathpazo}
\usepackage{a4wide}

\setmainfont[Ligatures=TeX]{TeX Gyre Pagella}

\title{Introduction à l'algorithmique}
\author{Lionel D. K\small{ITIHOUN}}

\begin{document}
\maketitle

\begin{abstract}
  Ce document est conçu pour servir de support au cours d'introduction à l'algorithmique et à la programmation. Il est destiné aux étudiants en première année d'informatique ou en classes préparatoires.
\end{abstract}


\chapter{Présentation d'un algorithme}
Un algorithme est comme un livre. On le lit de gauche à droite et de haut en base. Tout comme un livre, un algo comporte les éléments suivants:
\begin{itemize}
  \item un titre,
  \item une section pour les remerciements,
  \item un contenu,
  \item des notes de bas de page (commentaires).
\end{itemize}
\end{document}
